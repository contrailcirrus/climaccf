%% Generated by Sphinx.
\def\sphinxdocclass{report}
\documentclass[a4paper,11pt,english]{sphinxmanual}
\ifdefined\pdfpxdimen
   \let\sphinxpxdimen\pdfpxdimen\else\newdimen\sphinxpxdimen
\fi \sphinxpxdimen=.75bp\relax

\PassOptionsToPackage{warn}{textcomp}
\usepackage[utf8]{inputenc}
\ifdefined\DeclareUnicodeCharacter
% support both utf8 and utf8x syntaxes
  \ifdefined\DeclareUnicodeCharacterAsOptional
    \def\sphinxDUC#1{\DeclareUnicodeCharacter{"#1}}
  \else
    \let\sphinxDUC\DeclareUnicodeCharacter
  \fi
  \sphinxDUC{00A0}{\nobreakspace}
  \sphinxDUC{2500}{\sphinxunichar{2500}}
  \sphinxDUC{2502}{\sphinxunichar{2502}}
  \sphinxDUC{2514}{\sphinxunichar{2514}}
  \sphinxDUC{251C}{\sphinxunichar{251C}}
  \sphinxDUC{2572}{\textbackslash}
\fi
\usepackage{cmap}
\usepackage[T1]{fontenc}
\usepackage{amsmath,amssymb,amstext}
\usepackage{babel}



\usepackage{times}
\expandafter\ifx\csname T@LGR\endcsname\relax
\else
% LGR was declared as font encoding
  \substitutefont{LGR}{\rmdefault}{cmr}
  \substitutefont{LGR}{\sfdefault}{cmss}
  \substitutefont{LGR}{\ttdefault}{cmtt}
\fi
\expandafter\ifx\csname T@X2\endcsname\relax
  \expandafter\ifx\csname T@T2A\endcsname\relax
  \else
  % T2A was declared as font encoding
    \substitutefont{T2A}{\rmdefault}{cmr}
    \substitutefont{T2A}{\sfdefault}{cmss}
    \substitutefont{T2A}{\ttdefault}{cmtt}
  \fi
\else
% X2 was declared as font encoding
  \substitutefont{X2}{\rmdefault}{cmr}
  \substitutefont{X2}{\sfdefault}{cmss}
  \substitutefont{X2}{\ttdefault}{cmtt}
\fi


\usepackage[Bjarne]{fncychap}
\usepackage{sphinx}

\fvset{fontsize=\small}
\usepackage{geometry}


% Include hyperref last.
\usepackage{hyperref}
% Fix anchor placement for figures with captions.
\usepackage{hypcap}% it must be loaded after hyperref.
% Set up styles of URL: it should be placed after hyperref.
\urlstyle{same}

\addto\captionsenglish{\renewcommand{\contentsname}{Contents:}}

\usepackage{sphinxmessages}
\setcounter{tocdepth}{1}


    \usepackage
    {charter}
    \usepackage[defaultsans]
    {lato}
    \usepackage
    {inconsolata}
    

\title{Environmental Library Documentation}
\date{Nov 24, 2021}
\release{V0.0.1}
\author{UC3M, DLR}
\newcommand{\sphinxlogo}{\vbox{}}
\renewcommand{\releasename}{Release}
\makeindex
\begin{document}

\ifdefined\shorthandoff
  \ifnum\catcode`\=\string=\active\shorthandoff{=}\fi
  \ifnum\catcode`\"=\active\shorthandoff{"}\fi
\fi

\pagestyle{empty}
\sphinxmaketitle
\pagestyle{plain}
\sphinxtableofcontents
\pagestyle{normal}
\phantomsection\label{\detokenize{index::doc}}



\chapter{What is EnVLiB?}
\label{\detokenize{index:what-is-envlib}}
ENVLIB is a libray that calculates algorithmic climate change functions (aCCFs).
It is distributed under the GNU Lesser General Public License v3.0.

\sphinxstylestrong{Citation info}: A. Simorgh, M. Soler, D. Daniel González\sphinxhyphen{}Arribas, Environmental library (EnVLiB), an open source toolbox developed to calculate algorithmic climate change functions (aCCF), which quantifies climate impacts of aviation.


\chapter{How to run the library}
\label{\detokenize{index:how-to-run-the-library}}\begin{enumerate}
\sphinxsetlistlabels{\arabic}{enumi}{enumii}{}{.}%
\setcounter{enumi}{-1}
\item {} 
it is highly recomended to create a virtual environment with Python version 3.8:

\end{enumerate}

\begin{sphinxVerbatim}[commandchars=\\\{\}]
\PYG{n}{conda} \PYG{n}{create} \PYG{o}{\PYGZhy{}}\PYG{n}{n} \PYG{n}{name\PYGZus{}env} \PYG{n}{python}\PYG{o}{=}\PYG{l+m+mf}{3.8}
\PYG{n}{conda} \PYG{n}{activate} \PYG{n}{name\PYGZus{}env}
\end{sphinxVerbatim}
\begin{enumerate}
\sphinxsetlistlabels{\arabic}{enumi}{enumii}{}{.}%
\item {} 
Clone or download the repository.

\item {} 
Locate yourself in the envlib (library folder) path, and run the following line, using terminal (MacOS) or cmd (Windows), which will install all dependencies:

\end{enumerate}

\begin{sphinxVerbatim}[commandchars=\\\{\}]
\PYG{n}{python} \PYG{n}{setup}\PYG{o}{.}\PYG{n}{py} \PYG{n}{install}
\end{sphinxVerbatim}


\chapter{How to use it}
\label{\detokenize{index:how-to-use-it}}\begin{enumerate}
\sphinxsetlistlabels{\arabic}{enumi}{enumii}{}{.}%
\item {} 
import library:

\end{enumerate}
\begin{description}
\item[{::}] \leavevmode
import envlib,
from envlib import main\_processing

\end{description}
\begin{enumerate}
\sphinxsetlistlabels{\arabic}{enumi}{enumii}{}{.}%
\setcounter{enumi}{1}
\item {} 
Specify the directories for datasets containing variables on pressure levels and surface in a dictioary as:

\end{enumerate}

\begin{sphinxVerbatim}[commandchars=\\\{\}]
\PYG{n}{path} \PYG{o}{=} \PYG{p}{\PYGZob{}}\PYG{l+s+s1}{\PYGZsq{}}\PYG{l+s+s1}{path\PYGZus{}pl}\PYG{l+s+s1}{\PYGZsq{}}\PYG{p}{:} \PYG{n}{location\PYGZus{}pressure\PYGZus{}variables}\PYG{p}{,} \PYG{l+s+s1}{\PYGZsq{}}\PYG{l+s+s1}{path\PYGZus{}sur}\PYG{l+s+s1}{\PYGZsq{}}\PYG{p}{:} \PYG{n}{location\PYGZus{}surface\PYGZus{}variables}\PYG{l+s+s1}{\PYGZsq{}}\PYG{l+s+s1}{\PYGZcb{}}
\end{sphinxVerbatim}
\begin{enumerate}
\sphinxsetlistlabels{\arabic}{enumi}{enumii}{}{.}%
\setcounter{enumi}{2}
\item {} 
Specify the directory for the output file:

\end{enumerate}
\begin{description}
\item[{::}] \leavevmode
path\_save = location\_results

\end{description}
\begin{enumerate}
\sphinxsetlistlabels{\arabic}{enumi}{enumii}{}{.}%
\setcounter{enumi}{3}
\item {} \begin{description}
\item[{Set the preferred configurations in a dictionary:}] \leavevmode
Set the preferred configurations in a dictionary. Settings are:
\begin{itemize}
	\item  \textbf{efficacy}: True, False
	\item \textbf{emission\_scenario}: sustained, future\_scenario, pulse 
	\item \textbf{limate\_indicator}: ATR, GWP 
	\item \textbf{TimeHorizon}: 20, 50, 100 
	\item \textbf{PMO}: True, False 
	\item \textbf{merged}: True, False 
	\item \textbf{NOx}: True, False 
	\item \textbf{Chotspots}: True, False 
	\item \textbf{hotspots\_thr}: constant, varying 
	\item \textbf{binary}: True, False 
	\item \textbf{variables}: True, False 
	\item \textbf{mean}: True, False 
	\item \textbf{std}: True, False 
\end{itemize}



\end{description}

\end{enumerate}

\begin{sphinxVerbatim}[commandchars=\\\{\}]
\PYG{n}{confg} \PYG{o}{=} \PYG{p}{\PYGZob{}}\PYG{l+s+s1}{\PYGZsq{}}\PYG{l+s+s1}{efficacy}\PYG{l+s+s1}{\PYGZsq{}}\PYG{p}{:} \PYG{k+kc}{False}\PYG{p}{,} \PYG{l+s+s1}{\PYGZsq{}}\PYG{l+s+s1}{emission\PYGZus{}scenario}\PYG{l+s+s1}{\PYGZsq{}}\PYG{p}{:} \PYG{l+s+s1}{\PYGZsq{}}\PYG{l+s+s1}{pulse}\PYG{l+s+s1}{\PYGZsq{}}\PYG{p}{,} \PYG{l+s+s1}{\PYGZsq{}}\PYG{l+s+s1}{climate\PYGZus{}indicator}\PYG{l+s+s1}{\PYGZsq{}}\PYG{p}{:} \PYG{l+s+s1}{\PYGZsq{}}\PYG{l+s+s1}{ATR}\PYG{l+s+s1}{\PYGZsq{}}\PYG{p}{,} \PYG{l+s+s1}{\PYGZsq{}}\PYG{l+s+s1}{TimeHorizon}\PYG{l+s+s1}{\PYGZsq{}}\PYG{p}{:} \PYG{l+m+mi}{20}\PYG{p}{,}
         \PYG{l+s+s1}{\PYGZsq{}}\PYG{l+s+s1}{PMO}\PYG{l+s+s1}{\PYGZsq{}}\PYG{p}{:} \PYG{k+kc}{False}\PYG{p}{,}\PYG{l+s+s1}{\PYGZsq{}}\PYG{l+s+s1}{merged}\PYG{l+s+s1}{\PYGZsq{}}\PYG{p}{:} \PYG{k+kc}{True}\PYG{p}{,} \PYG{l+s+s1}{\PYGZsq{}}\PYG{l+s+s1}{NOx}\PYG{l+s+s1}{\PYGZsq{}}\PYG{p}{:} \PYG{k+kc}{True}\PYG{p}{,} \PYG{l+s+s1}{\PYGZsq{}}\PYG{l+s+s1}{Chotspots}\PYG{l+s+s1}{\PYGZsq{}}\PYG{p}{:} \PYG{k+kc}{True}\PYG{p}{,} \PYG{l+s+s1}{\PYGZsq{}}\PYG{l+s+s1}{binary}\PYG{l+s+s1}{\PYGZsq{}}\PYG{p}{:} \PYG{k+kc}{True}\PYG{p}{,}
         \PYG{l+s+s1}{\PYGZsq{}}\PYG{l+s+s1}{hotspots\PYGZus{}thr}\PYG{l+s+s1}{\PYGZsq{}}\PYG{p}{:} \PYG{l+m+mf}{1e\PYGZhy{}13}\PYG{p}{,} \PYG{l+s+s1}{\PYGZsq{}}\PYG{l+s+s1}{variables}\PYG{l+s+s1}{\PYGZsq{}}\PYG{p}{:} \PYG{k+kc}{False}\PYG{p}{,} \PYG{l+s+s1}{\PYGZsq{}}\PYG{l+s+s1}{mean}\PYG{l+s+s1}{\PYGZsq{}}\PYG{p}{:} \PYG{k+kc}{False}\PYG{p}{,} \PYG{l+s+s1}{\PYGZsq{}}\PYG{l+s+s1}{std}\PYG{l+s+s1}{\PYGZsq{}}\PYG{p}{:} \PYG{k+kc}{False}\PYG{p}{\PYGZcb{}}
\end{sphinxVerbatim}
\begin{enumerate}
\sphinxsetlistlabels{\arabic}{enumi}{enumii}{}{.}%
\setcounter{enumi}{4}
\item {} 
Process inputted data:

\end{enumerate}

\begin{sphinxVerbatim}[commandchars=\\\{\}]
\PYG{n}{CI} \PYG{o}{=} \PYG{n}{processing\PYGZus{}main}\PYG{o}{.}\PYG{n}{ClimateImpact}\PYG{p}{(}\PYG{n}{path}\PYG{p}{,} \PYG{n}{horizontal\PYGZus{}resolution}\PYG{o}{=}\PYG{n}{resolution}\PYG{p}{,} \PYG{n}{lat\PYGZus{}bound}\PYG{o}{=}\PYG{p}{(}\PYG{n}{lat\PYGZus{}min}\PYG{p}{,} \PYG{n}{lat\PYGZus{}max}\PYG{p}{)}\PYG{p}{,} \PYG{n}{lon\PYGZus{}bound}\PYG{o}{=}\PYG{p}{(}\PYG{n}{lon\PYGZus{}min}\PYG{p}{,} \PYG{n}{lon\PYGZus{}max}\PYG{p}{)}\PYG{p}{,}
                                      \PYG{n}{save\PYGZus{}path}\PYG{o}{=}\PYG{n}{path\PYGZus{}save}\PYG{p}{)}
\end{sphinxVerbatim}
\begin{enumerate}
\sphinxsetlistlabels{\arabic}{enumi}{enumii}{}{.}%
\setcounter{enumi}{5}
\item {} 
Calculate aCCFs with respect to the defined settings in the dictionary, Confg, and store the results in a netCDF file:

\end{enumerate}

\begin{sphinxVerbatim}[commandchars=\\\{\}]
\PYG{n}{CI}\PYG{o}{.}\PYG{n}{calculate\PYGZus{}accfs}\PYG{p}{(}\PYG{o}{*}\PYG{o}{*}\PYG{n}{confg}\PYG{p}{)}
\end{sphinxVerbatim}


\chapter{An example}
\label{\detokenize{index:an-example}}\begin{enumerate}
\sphinxsetlistlabels{\arabic}{enumi}{enumii}{}{.}%
\setcounter{enumi}{-1}
\item {} 
Here is an example how one can use sample data in test directory of envlib to generate output for a set of user\sphinxhyphen{}difned configurations:

\end{enumerate}

\begin{sphinxVerbatim}[commandchars=\\\{\}]
\PYG{k+kn}{import} \PYG{n+nn}{envlib}
\PYG{k+kn}{from} \PYG{n+nn}{envlib} \PYG{k+kn}{import} \PYG{n}{main\PYGZus{}processing}

\PYG{n}{path\PYGZus{}here} \PYG{o}{=} \PYG{l+s+s1}{\PYGZsq{}}\PYG{l+s+s1}{envlib/}\PYG{l+s+s1}{\PYGZsq{}}
\PYG{n}{test\PYGZus{}path} \PYG{o}{=} \PYG{n}{path\PYGZus{}here} \PYG{o}{+} \PYG{l+s+s1}{\PYGZsq{}}\PYG{l+s+s1}{/test/sample\PYGZus{}data/}\PYG{l+s+s1}{\PYGZsq{}}
\PYG{n}{path\PYGZus{}} \PYG{o}{=} \PYG{p}{\PYGZob{}}\PYG{l+s+s1}{\PYGZsq{}}\PYG{l+s+s1}{path\PYGZus{}pl}\PYG{l+s+s1}{\PYGZsq{}}\PYG{p}{:} \PYG{n}{test\PYGZus{}path} \PYG{o}{+} \PYG{l+s+s1}{\PYGZsq{}}\PYG{l+s+s1}{sample\PYGZus{}pl.nc}\PYG{l+s+s1}{\PYGZsq{}}\PYG{p}{,} \PYG{l+s+s1}{\PYGZsq{}}\PYG{l+s+s1}{path\PYGZus{}sur}\PYG{l+s+s1}{\PYGZsq{}}\PYG{p}{:} \PYG{n}{test\PYGZus{}path} \PYG{o}{+} \PYG{l+s+s1}{\PYGZsq{}}\PYG{l+s+s1}{sample\PYGZus{}sur.nc}\PYG{l+s+s1}{\PYGZsq{}}\PYG{p}{\PYGZcb{}}
\PYG{n}{path\PYGZus{}save} \PYG{o}{=} \PYG{n}{test\PYGZus{}path} \PYG{o}{+} \PYG{l+s+s1}{\PYGZsq{}}\PYG{l+s+s1}{env\PYGZus{}processed.nc}\PYG{l+s+s1}{\PYGZsq{}}
\PYG{n}{confg} \PYG{o}{=} \PYG{p}{\PYGZob{}}\PYG{l+s+s1}{\PYGZsq{}}\PYG{l+s+s1}{efficacy}\PYG{l+s+s1}{\PYGZsq{}}\PYG{p}{:} \PYG{k+kc}{False}\PYG{p}{,} \PYG{l+s+s1}{\PYGZsq{}}\PYG{l+s+s1}{emission\PYGZus{}scenario}\PYG{l+s+s1}{\PYGZsq{}}\PYG{p}{:} \PYG{l+s+s1}{\PYGZsq{}}\PYG{l+s+s1}{pulse}\PYG{l+s+s1}{\PYGZsq{}}\PYG{p}{,} \PYG{l+s+s1}{\PYGZsq{}}\PYG{l+s+s1}{climate\PYGZus{}indicator}\PYG{l+s+s1}{\PYGZsq{}}\PYG{p}{:} \PYG{l+s+s1}{\PYGZsq{}}\PYG{l+s+s1}{ATR}\PYG{l+s+s1}{\PYGZsq{}}\PYG{p}{,} \PYG{l+s+s1}{\PYGZsq{}}\PYG{l+s+s1}{TimeHorizon}\PYG{l+s+s1}{\PYGZsq{}}\PYG{p}{:} \PYG{l+m+mi}{20}\PYG{p}{,}
             \PYG{l+s+s1}{\PYGZsq{}}\PYG{l+s+s1}{PMO}\PYG{l+s+s1}{\PYGZsq{}}\PYG{p}{:} \PYG{k+kc}{False}\PYG{p}{,}
             \PYG{l+s+s1}{\PYGZsq{}}\PYG{l+s+s1}{merged}\PYG{l+s+s1}{\PYGZsq{}}\PYG{p}{:} \PYG{k+kc}{True}\PYG{p}{,} \PYG{l+s+s1}{\PYGZsq{}}\PYG{l+s+s1}{NOx}\PYG{l+s+s1}{\PYGZsq{}}\PYG{p}{:} \PYG{k+kc}{True}\PYG{p}{,} \PYG{l+s+s1}{\PYGZsq{}}\PYG{l+s+s1}{Chotspots}\PYG{l+s+s1}{\PYGZsq{}}\PYG{p}{:} \PYG{k+kc}{True}\PYG{p}{,} \PYG{l+s+s1}{\PYGZsq{}}\PYG{l+s+s1}{binary}\PYG{l+s+s1}{\PYGZsq{}}\PYG{p}{:} \PYG{k+kc}{True}\PYG{p}{,}
             \PYG{l+s+s1}{\PYGZsq{}}\PYG{l+s+s1}{hotspots\PYGZus{}thr}\PYG{l+s+s1}{\PYGZsq{}}\PYG{p}{:} \PYG{l+m+mf}{1e\PYGZhy{}13}\PYG{p}{,} \PYG{l+s+s1}{\PYGZsq{}}\PYG{l+s+s1}{variables}\PYG{l+s+s1}{\PYGZsq{}}\PYG{p}{:} \PYG{k+kc}{False}\PYG{p}{,} \PYG{l+s+s1}{\PYGZsq{}}\PYG{l+s+s1}{mean}\PYG{l+s+s1}{\PYGZsq{}}\PYG{p}{:} \PYG{k+kc}{False}\PYG{p}{,} \PYG{l+s+s1}{\PYGZsq{}}\PYG{l+s+s1}{std}\PYG{l+s+s1}{\PYGZsq{}}\PYG{p}{:} \PYG{k+kc}{False}\PYG{p}{\PYGZcb{}}

\PYG{n}{CI} \PYG{o}{=} \PYG{n}{main\PYGZus{}processing}\PYG{o}{.}\PYG{n}{ClimateImpact}\PYG{p}{(}\PYG{n}{path\PYGZus{}}\PYG{p}{,} \PYG{n}{horizontal\PYGZus{}resolution}\PYG{o}{=}\PYG{l+m+mf}{0.75}\PYG{p}{,} \PYG{n}{lat\PYGZus{}bound}\PYG{o}{=}\PYG{p}{(}\PYG{l+m+mi}{35}\PYG{p}{,} \PYG{l+m+mf}{60.0}\PYG{p}{)}\PYG{p}{,} \PYG{n}{lon\PYGZus{}bound}\PYG{o}{=}\PYG{p}{(}\PYG{o}{\PYGZhy{}}\PYG{l+m+mi}{15}\PYG{p}{,} \PYG{l+m+mi}{35}\PYG{p}{)}\PYG{p}{,}
                                          \PYG{n}{save\PYGZus{}path}\PYG{o}{=}\PYG{n}{path\PYGZus{}save}\PYG{p}{)}
\PYG{n}{CI}\PYG{o}{.}\PYG{n}{calculate\PYGZus{}accfs}\PYG{p}{(}\PYG{o}{*}\PYG{o}{*}\PYG{n}{confg}\PYG{p}{)}
\end{sphinxVerbatim}


\chapter{How to compile documentation pdf?}
\label{\detokenize{index:how-to-compile-documentation-pdf}}
You can use the Makefile created by Sphinx to create your documentation. Locate yourself in the doc path.

First clean the \_build directory to avoid error or legacy information. Just call:

\begin{sphinxVerbatim}[commandchars=\\\{\}]
\PYG{n}{make} \PYG{n}{clean}
\end{sphinxVerbatim}

In case you want to build your documentation in latex call \sphinxstylestrong{twice}:

\begin{sphinxVerbatim}[commandchars=\\\{\}]
\PYG{n}{make} \PYG{n}{latexpdf}
\end{sphinxVerbatim}

if you want to do build your in html call:

\begin{sphinxVerbatim}[commandchars=\\\{\}]
\PYG{n}{make} \PYG{n}{html}
\end{sphinxVerbatim}

Note that you \sphinxstylestrong{should not see} any error or warning, this information appears as red text in the terminal.


\chapter{Modules:}
\label{\detokenize{index:modules}}

\section{envlib}
\label{\detokenize{modules:envlib}}\label{\detokenize{modules::doc}}

\subsection{envlib package}
\label{\detokenize{envlib:envlib-package}}\label{\detokenize{envlib::doc}}

\subsubsection{Submodules}
\label{\detokenize{envlib:submodules}}

\subsubsection{envlib.accf module}
\label{\detokenize{envlib:module-envlib.accf}}\label{\detokenize{envlib:envlib-accf-module}}\index{module@\spxentry{module}!envlib.accf@\spxentry{envlib.accf}}\index{envlib.accf@\spxentry{envlib.accf}!module@\spxentry{module}}\index{CalAccf (class in envlib.accf)@\spxentry{CalAccf}\spxextra{class in envlib.accf}}

\begin{fulllineitems}
\phantomsection\label{\detokenize{envlib:envlib.accf.CalAccf}}\pysiglinewithargsret{\sphinxbfcode{\sphinxupquote{class }}\sphinxcode{\sphinxupquote{envlib.accf.}}\sphinxbfcode{\sphinxupquote{CalAccf}}}{\emph{\DUrole{n}{wd\_inf}}}{}
Bases: \sphinxcode{\sphinxupquote{object}}

Calculation of algorithmic climate change functions (aCCFs).
\index{accf\_ch4() (envlib.accf.CalAccf method)@\spxentry{accf\_ch4()}\spxextra{envlib.accf.CalAccf method}}

\begin{fulllineitems}
\phantomsection\label{\detokenize{envlib:envlib.accf.CalAccf.accf_ch4}}\pysiglinewithargsret{\sphinxbfcode{\sphinxupquote{accf\_ch4}}}{}{}
Calculates the aCCF of Methane for pulse emission scenario, average temperature response as climate
indicator and 20 years (P\sphinxhyphen{}ATR20\sphinxhyphen{}methane {[}K/kg(NO2){]}). To calculate the aCCF of Methane, meteorological
variables geopotential and incoming solar radiation are required.
\begin{quote}\begin{description}
\item[{Returns accf}] \leavevmode
Algorithmic climate change function of methane.

\item[{Return type}] \leavevmode
numpy.ndarray

\end{description}\end{quote}

\end{fulllineitems}

\index{accf\_dcontrail() (envlib.accf.CalAccf method)@\spxentry{accf\_dcontrail()}\spxextra{envlib.accf.CalAccf method}}

\begin{fulllineitems}
\phantomsection\label{\detokenize{envlib:envlib.accf.CalAccf.accf_dcontrail}}\pysiglinewithargsret{\sphinxbfcode{\sphinxupquote{accf\_dcontrail}}}{}{}
Calculates the aCCF of day\sphinxhyphen{}time contrails for pulse emission scenario, average temperature response as
climate indicator and 20 years (P\sphinxhyphen{}ATR20\sphinxhyphen{}contrails {[}K/km{]}). To calculate the aCCF of day\sphinxhyphen{}time contrails,
meteorological variables ourgoing longwave radiation, temperature and relative humidities over ice and water
are required. Notice that, temperature and relative humidies are required for the detemiation of presistent
contrial formation areas.
\begin{quote}\begin{description}
\item[{Returns accf}] \leavevmode
Algorithmic climate change function of day\sphinxhyphen{}time contrails.

\item[{Return type}] \leavevmode
numpy.ndarray

\end{description}\end{quote}

\end{fulllineitems}

\index{accf\_h2o() (envlib.accf.CalAccf method)@\spxentry{accf\_h2o()}\spxextra{envlib.accf.CalAccf method}}

\begin{fulllineitems}
\phantomsection\label{\detokenize{envlib:envlib.accf.CalAccf.accf_h2o}}\pysiglinewithargsret{\sphinxbfcode{\sphinxupquote{accf\_h2o}}}{}{}
Calculates the aCCF of water vapour for pulse emission scenario, average temperature response as
climate indicator and 20 years (P\sphinxhyphen{}ATR20\sphinxhyphen{}water\sphinxhyphen{}vapour {[}K/kg(fuel){]}). To calculate the aCCF of water vapour,
meteorological variable potential vorticity is required.
\begin{quote}\begin{description}
\item[{Returns accf}] \leavevmode
Algorithmic climate change function of water vapour.

\item[{Return type}] \leavevmode
numpy.ndarray

\end{description}\end{quote}

\end{fulllineitems}

\index{accf\_ncontrail() (envlib.accf.CalAccf method)@\spxentry{accf\_ncontrail()}\spxextra{envlib.accf.CalAccf method}}

\begin{fulllineitems}
\phantomsection\label{\detokenize{envlib:envlib.accf.CalAccf.accf_ncontrail}}\pysiglinewithargsret{\sphinxbfcode{\sphinxupquote{accf\_ncontrail}}}{}{}
Calculates the aCCF of night\sphinxhyphen{}time contrails for pulse emission scenario, average temperature response as
climate indicator and 20 years (P\sphinxhyphen{}ATR20\sphinxhyphen{}contrails {[}K/km{]}). To calculate the aCCF of nighttime contrails,
meteorological variables temperature and relative humidities over ice and water are required. Notice that,
relative humidies are required for the detemiation of presistent contrial formation areas.
\begin{quote}\begin{description}
\item[{Returns accf}] \leavevmode
Algorithmic climate change function of nighttime contrails.

\item[{Return type}] \leavevmode
numpy.ndarray

\end{description}\end{quote}

\end{fulllineitems}

\index{accf\_o3() (envlib.accf.CalAccf method)@\spxentry{accf\_o3()}\spxextra{envlib.accf.CalAccf method}}

\begin{fulllineitems}
\phantomsection\label{\detokenize{envlib:envlib.accf.CalAccf.accf_o3}}\pysiglinewithargsret{\sphinxbfcode{\sphinxupquote{accf\_o3}}}{}{}
Calculates the aCCF of Ozone for pulse emission scenario, average temperature response as climate
indicator and 20 years (P\sphinxhyphen{}ATR20\sphinxhyphen{}ozone {[}K/kg(NO2){]}). To calculate the aCCF of Ozone, meteorological variables
temperature and geopotential are required.
\begin{quote}\begin{description}
\item[{Returns accf}] \leavevmode
Algorithmic climate change function of Ozone.

\item[{Return type}] \leavevmode
numpy.ndarray

\end{description}\end{quote}

\end{fulllineitems}

\index{get\_accfs() (envlib.accf.CalAccf method)@\spxentry{get\_accfs()}\spxextra{envlib.accf.CalAccf method}}

\begin{fulllineitems}
\phantomsection\label{\detokenize{envlib:envlib.accf.CalAccf.get_accfs}}\pysiglinewithargsret{\sphinxbfcode{\sphinxupquote{get\_accfs}}}{\emph{\DUrole{o}{**}\DUrole{n}{problem\_config}}}{}
Gets the formulations of aCCFs, and calculated user\sphinxhyphen{}defined conversions or functions.

\end{fulllineitems}

\index{get\_std() (envlib.accf.CalAccf method)@\spxentry{get\_std()}\spxextra{envlib.accf.CalAccf method}}

\begin{fulllineitems}
\phantomsection\label{\detokenize{envlib:envlib.accf.CalAccf.get_std}}\pysiglinewithargsret{\sphinxbfcode{\sphinxupquote{get\_std}}}{\emph{\DUrole{n}{var}}, \emph{\DUrole{n}{normalize}\DUrole{o}{=}\DUrole{default_value}{False}}}{}
Calculates standard deviation of a variable over ensemble members.
\begin{quote}\begin{description}
\item[{Parameters}] \leavevmode\begin{itemize}
\item {} 
\sphinxstyleliteralstrong{\sphinxupquote{var}} \textendash{} variable.

\item {} 
\sphinxstyleliteralstrong{\sphinxupquote{normalize}} \textendash{} If True, it calculated standard deviation over the normalized variable, if False,

\end{itemize}

\item[{Return type}] \leavevmode
numpy.ndarray

\end{description}\end{quote}

from the original variable.
:rtype: bool

:returns standard deviation of the variable.
:rtype: numpy.ndarray

\end{fulllineitems}

\index{get\_xarray() (envlib.accf.CalAccf method)@\spxentry{get\_xarray()}\spxextra{envlib.accf.CalAccf method}}

\begin{fulllineitems}
\phantomsection\label{\detokenize{envlib:envlib.accf.CalAccf.get_xarray}}\pysiglinewithargsret{\sphinxbfcode{\sphinxupquote{get\_xarray}}}{}{}
Build xarray dataset.
\begin{quote}\begin{description}
\item[{Returns ds}] \leavevmode
xarray dataset containing user\sphinxhyphen{}difned variables (e.g., merged aCCFs, mean aCCFs, Climate hotspots).

\item[{Return type}] \leavevmode
dataset

\end{description}\end{quote}

:returns encoding
:rtype: dict

\end{fulllineitems}


\end{fulllineitems}

\index{convert\_accf() (in module envlib.accf)@\spxentry{convert\_accf()}\spxextra{in module envlib.accf}}

\begin{fulllineitems}
\phantomsection\label{\detokenize{envlib:envlib.accf.convert_accf}}\pysiglinewithargsret{\sphinxcode{\sphinxupquote{envlib.accf.}}\sphinxbfcode{\sphinxupquote{convert\_accf}}}{\emph{\DUrole{n}{name}}, \emph{\DUrole{n}{value}}, \emph{\DUrole{n}{confg}}}{}
\end{fulllineitems}

\index{get\_Fin() (in module envlib.accf)@\spxentry{get\_Fin()}\spxextra{in module envlib.accf}}

\begin{fulllineitems}
\phantomsection\label{\detokenize{envlib:envlib.accf.get_Fin}}\pysiglinewithargsret{\sphinxcode{\sphinxupquote{envlib.accf.}}\sphinxbfcode{\sphinxupquote{get\_Fin}}}{\emph{\DUrole{n}{ds}}, \emph{\DUrole{n}{lat}}}{}
\end{fulllineitems}

\index{get\_encoding\_dict() (in module envlib.accf)@\spxentry{get\_encoding\_dict()}\spxextra{in module envlib.accf}}

\begin{fulllineitems}
\phantomsection\label{\detokenize{envlib:envlib.accf.get_encoding_dict}}\pysiglinewithargsret{\sphinxcode{\sphinxupquote{envlib.accf.}}\sphinxbfcode{\sphinxupquote{get\_encoding\_dict}}}{\emph{\DUrole{n}{list\_name}}, \emph{\DUrole{n}{encoding}}}{}
\end{fulllineitems}

\index{get\_latlon() (in module envlib.accf)@\spxentry{get\_latlon()}\spxextra{in module envlib.accf}}

\begin{fulllineitems}
\phantomsection\label{\detokenize{envlib:envlib.accf.get_latlon}}\pysiglinewithargsret{\sphinxcode{\sphinxupquote{envlib.accf.}}\sphinxbfcode{\sphinxupquote{get\_latlon}}}{\emph{\DUrole{n}{ds}}, \emph{\DUrole{n}{member\_bool}}}{}
\end{fulllineitems}



\subsubsection{envlib.calc\_altrv\_vars module}
\label{\detokenize{envlib:module-envlib.calc_altrv_vars}}\label{\detokenize{envlib:envlib-calc-altrv-vars-module}}\index{module@\spxentry{module}!envlib.calc\_altrv\_vars@\spxentry{envlib.calc\_altrv\_vars}}\index{envlib.calc\_altrv\_vars@\spxentry{envlib.calc\_altrv\_vars}!module@\spxentry{module}}\index{get\_pvu() (in module envlib.calc\_altrv\_vars)@\spxentry{get\_pvu()}\spxextra{in module envlib.calc\_altrv\_vars}}

\begin{fulllineitems}
\phantomsection\label{\detokenize{envlib:envlib.calc_altrv_vars.get_pvu}}\pysiglinewithargsret{\sphinxcode{\sphinxupquote{envlib.calc\_altrv\_vars.}}\sphinxbfcode{\sphinxupquote{get\_pvu}}}{\emph{\DUrole{n}{ds}}}{}
Caclulates potential vorticity from meteorological variables temperature and components of winds.
\begin{quote}\begin{description}
\item[{Parameters}] \leavevmode
\sphinxstyleliteralstrong{\sphinxupquote{ds}} (\sphinxstyleliteralemphasis{\sphinxupquote{Dataset}}) \textendash{} Dataset openned with xarray.

\item[{Returns PVUU}] \leavevmode
potential vorticity unit

\item[{Return type}] \leavevmode
numpy.ndarray

\end{description}\end{quote}

\end{fulllineitems}

\index{get\_r() (in module envlib.calc\_altrv\_vars)@\spxentry{get\_r()}\spxextra{in module envlib.calc\_altrv\_vars}}

\begin{fulllineitems}
\phantomsection\label{\detokenize{envlib:envlib.calc_altrv_vars.get_r}}\pysiglinewithargsret{\sphinxcode{\sphinxupquote{envlib.calc\_altrv\_vars.}}\sphinxbfcode{\sphinxupquote{get\_r}}}{\emph{\DUrole{n}{ds}}}{}
\end{fulllineitems}

\index{get\_rh\_ice() (in module envlib.calc\_altrv\_vars)@\spxentry{get\_rh\_ice()}\spxextra{in module envlib.calc\_altrv\_vars}}

\begin{fulllineitems}
\phantomsection\label{\detokenize{envlib:envlib.calc_altrv_vars.get_rh_ice}}\pysiglinewithargsret{\sphinxcode{\sphinxupquote{envlib.calc\_altrv\_vars.}}\sphinxbfcode{\sphinxupquote{get\_rh\_ice}}}{\emph{\DUrole{n}{ds}}}{}
Calculates the relative humidity over ice from realtive humidity over water
\begin{quote}\begin{description}
\item[{Parameters}] \leavevmode
\sphinxstyleliteralstrong{\sphinxupquote{ds}} (\sphinxstyleliteralemphasis{\sphinxupquote{Dataset}}) \textendash{} Dataset openned with xarray.

\item[{Returns rh\_ice}] \leavevmode
relative humidity over ice

\item[{Return type}] \leavevmode
numpy.ndarray

\end{description}\end{quote}

\end{fulllineitems}

\index{get\_rh\_sd() (in module envlib.calc\_altrv\_vars)@\spxentry{get\_rh\_sd()}\spxextra{in module envlib.calc\_altrv\_vars}}

\begin{fulllineitems}
\phantomsection\label{\detokenize{envlib:envlib.calc_altrv_vars.get_rh_sd}}\pysiglinewithargsret{\sphinxcode{\sphinxupquote{envlib.calc\_altrv\_vars.}}\sphinxbfcode{\sphinxupquote{get\_rh\_sd}}}{\emph{\DUrole{n}{ds}}}{}
Calculates the relative humidity from specific humidity
\begin{quote}\begin{description}
\item[{Parameters}] \leavevmode
\sphinxstyleliteralstrong{\sphinxupquote{ds}} (\sphinxstyleliteralemphasis{\sphinxupquote{Dataset}}) \textendash{} Dataset openned with xarray.

\item[{Returns rh\_sd}] \leavevmode
relative humidity

\item[{Return type}] \leavevmode
numpy.ndarray

\end{description}\end{quote}

\end{fulllineitems}

\index{get\_rh\_wa() (in module envlib.calc\_altrv\_vars)@\spxentry{get\_rh\_wa()}\spxextra{in module envlib.calc\_altrv\_vars}}

\begin{fulllineitems}
\phantomsection\label{\detokenize{envlib:envlib.calc_altrv_vars.get_rh_wa}}\pysiglinewithargsret{\sphinxcode{\sphinxupquote{envlib.calc\_altrv\_vars.}}\sphinxbfcode{\sphinxupquote{get\_rh\_wa}}}{\emph{\DUrole{n}{ds}}}{}
Calculates the relative humidity over water from specific humidity
\begin{quote}\begin{description}
\item[{Parameters}] \leavevmode
\sphinxstyleliteralstrong{\sphinxupquote{ds}} (\sphinxstyleliteralemphasis{\sphinxupquote{Dataset}}) \textendash{} Dataset openned with xarray.

\item[{Returns rh\_wa}] \leavevmode
relative humidity over water

\item[{Return type}] \leavevmode
numpy.ndarray

\end{description}\end{quote}

\end{fulllineitems}



\subsubsection{envlib.contrail module}
\label{\detokenize{envlib:module-envlib.contrail}}\label{\detokenize{envlib:envlib-contrail-module}}\index{module@\spxentry{module}!envlib.contrail@\spxentry{envlib.contrail}}\index{envlib.contrail@\spxentry{envlib.contrail}!module@\spxentry{module}}\index{get\_cont\_form\_thr() (in module envlib.contrail)@\spxentry{get\_cont\_form\_thr()}\spxextra{in module envlib.contrail}}

\begin{fulllineitems}
\phantomsection\label{\detokenize{envlib:envlib.contrail.get_cont_form_thr}}\pysiglinewithargsret{\sphinxcode{\sphinxupquote{envlib.contrail.}}\sphinxbfcode{\sphinxupquote{get\_cont\_form\_thr}}}{\emph{\DUrole{n}{ds}}, \emph{\DUrole{n}{member}}}{}
Calculates the thresholds of temperature and relative humidity over water needed for determining the
formation criteria of contrails.
\begin{quote}
\begin{quote}\begin{description}
\item[{param ds}] \leavevmode
Dataset openned with xarray.

\item[{type ds}] \leavevmode
Dataset

\item[{param member}] \leavevmode
Detemines the presense of ensemble forecasts in the given dataset.

\item[{type member}] \leavevmode
bool

\item[{returns rcontr}] \leavevmode
Thresholds of relative humidity over water.

\item[{rtype}] \leavevmode
numpy.ndarray

\item[{returns TLM\_e}] \leavevmode
Thresholds of temperature.

\item[{rtype}] \leavevmode
numpy.ndarray

\end{description}\end{quote}
\end{quote}

\end{fulllineitems}

\index{get\_pcfa() (in module envlib.contrail)@\spxentry{get\_pcfa()}\spxextra{in module envlib.contrail}}

\begin{fulllineitems}
\phantomsection\label{\detokenize{envlib:envlib.contrail.get_pcfa}}\pysiglinewithargsret{\sphinxcode{\sphinxupquote{envlib.contrail.}}\sphinxbfcode{\sphinxupquote{get\_pcfa}}}{\emph{\DUrole{n}{ds}}, \emph{\DUrole{n}{member}}, \emph{\DUrole{o}{**}\DUrole{n}{problem\_config}}}{}
Calculates the presistent contrail formation areas (pcfa) which is used to calculate aCCF of (day/night) contrails.
\begin{quote}\begin{description}
\item[{Parameters}] \leavevmode\begin{itemize}
\item {} 
\sphinxstyleliteralstrong{\sphinxupquote{ds}} (\sphinxstyleliteralemphasis{\sphinxupquote{Dataset}}) \textendash{} Dataset openned with xarray.

\item {} 
\sphinxstyleliteralstrong{\sphinxupquote{member}} (\sphinxstyleliteralemphasis{\sphinxupquote{bool}}) \textendash{} Detemines the presense of ensemble forecasts in the given dataset.

\end{itemize}

\item[{Returns pcfa{\color{red}\bfseries{}}}] \leavevmode
Presistent contrail formation areas.

\item[{Return type}] \leavevmode
numpy.ndarray

\end{description}\end{quote}

\end{fulllineitems}

\index{get\_relative\_hum() (in module envlib.contrail)@\spxentry{get\_relative\_hum()}\spxextra{in module envlib.contrail}}

\begin{fulllineitems}
\phantomsection\label{\detokenize{envlib:envlib.contrail.get_relative_hum}}\pysiglinewithargsret{\sphinxcode{\sphinxupquote{envlib.contrail.}}\sphinxbfcode{\sphinxupquote{get\_relative\_hum}}}{\emph{\DUrole{n}{ds}}, \emph{\DUrole{n}{member}}, \emph{\DUrole{n}{intrp}\DUrole{o}{=}\DUrole{default_value}{True}}}{}
Calculates the relative humidities over ice and water from the provided relative humidity within ECMWF
dataset. In ECMWF data: Relative humidity is defined with respect to saturation of the mixed phase: i.e. with
respect to saturation over ice below \sphinxhyphen{}23C and with respect to saturation over water above 0C. In the regime in
between a quadratic interpolation is applied.
\begin{quote}
\begin{quote}\begin{description}
\item[{param ds}] \leavevmode
Dataset openned with xarray.

\item[{type ds}] \leavevmode
Dataset

\item[{param member}] \leavevmode
Detemines the presense of ensemble forecasts in the given dataset.

\item[{type member}] \leavevmode
bool

\item[{returns rcontr}] \leavevmode
Thresholds of relative humidity over water.

\item[{rtype}] \leavevmode
numpy.ndarray

\item[{returns TLM\_e}] \leavevmode
Thresholds of temperature.

\item[{rtype}] \leavevmode
numpy.ndarray

\end{description}\end{quote}
\end{quote}

\end{fulllineitems}

\index{get\_rw\_from\_specific\_hum() (in module envlib.contrail)@\spxentry{get\_rw\_from\_specific\_hum()}\spxextra{in module envlib.contrail}}

\begin{fulllineitems}
\phantomsection\label{\detokenize{envlib:envlib.contrail.get_rw_from_specific_hum}}\pysiglinewithargsret{\sphinxcode{\sphinxupquote{envlib.contrail.}}\sphinxbfcode{\sphinxupquote{get\_rw\_from\_specific\_hum}}}{\emph{\DUrole{n}{ds}}, \emph{\DUrole{n}{member}}}{}
Calculates of relative humidity over water from specific humidity.
\begin{quote}\begin{description}
\item[{Parameters}] \leavevmode\begin{itemize}
\item {} 
\sphinxstyleliteralstrong{\sphinxupquote{ds}} (\sphinxstyleliteralemphasis{\sphinxupquote{Dataset}}) \textendash{} Dataset openned with xarray.

\item {} 
\sphinxstyleliteralstrong{\sphinxupquote{member}} (\sphinxstyleliteralemphasis{\sphinxupquote{bool}}) \textendash{} Detemines the presense of ensemble forecasts in the given dataset.

\end{itemize}

\item[{Returns r\_w}] \leavevmode
Relative humidity over water.

\item[{Return type}] \leavevmode
numpy.ndarray

\end{description}\end{quote}

\end{fulllineitems}

\index{potential\_con\_cir\_cov() (in module envlib.contrail)@\spxentry{potential\_con\_cir\_cov()}\spxextra{in module envlib.contrail}}

\begin{fulllineitems}
\phantomsection\label{\detokenize{envlib:envlib.contrail.potential_con_cir_cov}}\pysiglinewithargsret{\sphinxcode{\sphinxupquote{envlib.contrail.}}\sphinxbfcode{\sphinxupquote{potential\_con\_cir\_cov}}}{\emph{\DUrole{n}{ds}}, \emph{\DUrole{n}{r\_crit}}, \emph{\DUrole{n}{r\_ice}}}{}
\end{fulllineitems}



\subsubsection{envlib.database module}
\label{\detokenize{envlib:module-envlib.database}}\label{\detokenize{envlib:envlib-database-module}}\index{module@\spxentry{module}!envlib.database@\spxentry{envlib.database}}\index{envlib.database@\spxentry{envlib.database}!module@\spxentry{module}}

\subsubsection{envlib.emission\_indices module}
\label{\detokenize{envlib:module-envlib.emission_indices}}\label{\detokenize{envlib:envlib-emission-indices-module}}\index{module@\spxentry{module}!envlib.emission\_indices@\spxentry{envlib.emission\_indices}}\index{envlib.emission\_indices@\spxentry{envlib.emission\_indices}!module@\spxentry{module}}

\subsubsection{envlib.extend\_dim module}
\label{\detokenize{envlib:module-envlib.extend_dim}}\label{\detokenize{envlib:envlib-extend-dim-module}}\index{module@\spxentry{module}!envlib.extend\_dim@\spxentry{envlib.extend\_dim}}\index{envlib.extend\_dim@\spxentry{envlib.extend\_dim}!module@\spxentry{module}}\index{extend\_dimensions() (in module envlib.extend\_dim)@\spxentry{extend\_dimensions()}\spxextra{in module envlib.extend\_dim}}

\begin{fulllineitems}
\phantomsection\label{\detokenize{envlib:envlib.extend_dim.extend_dimensions}}\pysiglinewithargsret{\sphinxcode{\sphinxupquote{envlib.extend\_dim.}}\sphinxbfcode{\sphinxupquote{extend\_dimensions}}}{\emph{\DUrole{n}{inf\_coord}}, \emph{\DUrole{n}{ds}}, \emph{\DUrole{n}{ds\_sur}}, \emph{\DUrole{n}{ex\_variables}}}{}
Unifies the dimension of all types of given data as either 4\sphinxhyphen{}dimensional or 5\sphinxhyphen{}dimensional arrays, depending on
the exsitance of ensemple members. e.g., for a data only has two fields, latitude and longitude, this function
addds time and level fields.
\begin{quote}\begin{description}
\item[{Parameters}] \leavevmode\begin{itemize}
\item {} 
\sphinxstyleliteralstrong{\sphinxupquote{ds}} (\sphinxstyleliteralemphasis{\sphinxupquote{Dataset}}) \textendash{} information on original coordinates.

\item {} 
\sphinxstyleliteralstrong{\sphinxupquote{ds}} \textendash{} Dataset openned with xarray containg variables on pressure levels.

\item {} 
\sphinxstyleliteralstrong{\sphinxupquote{ds\_sur}} (\sphinxstyleliteralemphasis{\sphinxupquote{Dataset}}) \textendash{} Dataset containing surface parameters openned with xarray.

\item {} 
\sphinxstyleliteralstrong{\sphinxupquote{inf\_coord}} \textendash{} new coordinates

\end{itemize}

\item[{Returns ds\_pl}] \leavevmode
new dataset of pressure level variables regarding the added coordinates

\item[{Return type}] \leavevmode
dataset

\item[{Returns ds\_surf}] \leavevmode
new dataset of surface parameters regarding the added coordinates

\item[{Return type}] \leavevmode
dataset

\end{description}\end{quote}

\end{fulllineitems}



\subsubsection{envlib.extract\_data module}
\label{\detokenize{envlib:module-envlib.extract_data}}\label{\detokenize{envlib:envlib-extract-data-module}}\index{module@\spxentry{module}!envlib.extract\_data@\spxentry{envlib.extract\_data}}\index{envlib.extract\_data@\spxentry{envlib.extract\_data}!module@\spxentry{module}}\index{extract\_coordinates() (in module envlib.extract\_data)@\spxentry{extract\_coordinates()}\spxextra{in module envlib.extract\_data}}

\begin{fulllineitems}
\phantomsection\label{\detokenize{envlib:envlib.extract_data.extract_coordinates}}\pysiglinewithargsret{\sphinxcode{\sphinxupquote{envlib.extract\_data.}}\sphinxbfcode{\sphinxupquote{extract\_coordinates}}}{\emph{\DUrole{n}{ds}}, \emph{\DUrole{n}{ex\_variables}}, \emph{\DUrole{n}{ds\_sur}\DUrole{o}{=}\DUrole{default_value}{None}}}{}
Extract coordinates (axes) in the dataset defined with different possible names.
\begin{quote}\begin{description}
\item[{Parameters}] \leavevmode\begin{itemize}
\item {} 
\sphinxstyleliteralstrong{\sphinxupquote{ds\_sur}} \textendash{} 

\item {} 
\sphinxstyleliteralstrong{\sphinxupquote{ds}} (\sphinxstyleliteralemphasis{\sphinxupquote{Dataset}}) \textendash{} Dataset openned with xarray.

\end{itemize}

\item[{Returns ex\_var\_name}] \leavevmode
List of available coordinates.

\item[{Return type}] \leavevmode
list

\item[{Returns variables}] \leavevmode
Assigns bool to the axes (e.g., if ensmeble members are not available, it assgins False).

\item[{Return type}] \leavevmode
dict

\end{description}\end{quote}

\end{fulllineitems}

\index{extract\_data\_variables() (in module envlib.extract\_data)@\spxentry{extract\_data\_variables()}\spxextra{in module envlib.extract\_data}}

\begin{fulllineitems}
\phantomsection\label{\detokenize{envlib:envlib.extract_data.extract_data_variables}}\pysiglinewithargsret{\sphinxcode{\sphinxupquote{envlib.extract\_data.}}\sphinxbfcode{\sphinxupquote{extract\_data\_variables}}}{\emph{\DUrole{n}{ds}}, \emph{\DUrole{n}{ds\_sr}\DUrole{o}{=}\DUrole{default_value}{None}}, \emph{\DUrole{n}{verbose}\DUrole{o}{=}\DUrole{default_value}{False}}}{}
Extract available required variables in the dataset defined with different possible names.
\begin{quote}\begin{description}
\item[{Parameters}] \leavevmode\begin{itemize}
\item {} 
\sphinxstyleliteralstrong{\sphinxupquote{ds}} (\sphinxstyleliteralemphasis{\sphinxupquote{Dataset}}) \textendash{} Dataset openned with xarray.

\item {} 
\sphinxstyleliteralstrong{\sphinxupquote{ds\_sr}} (\sphinxstyleliteralemphasis{\sphinxupquote{Dataset}}) \textendash{} Dataset containing surface parameters openned with xarray.

\item {} 
\sphinxstyleliteralstrong{\sphinxupquote{verbose}} (\sphinxstyleliteralemphasis{\sphinxupquote{bool}}) \textendash{} Used to show more information.

\end{itemize}

\item[{Returns ex\_var\_name}] \leavevmode
Available required weather variables.

\item[{Return type}] \leavevmode
list

\item[{Returns variables}] \leavevmode
Assigns bool to the required wethear variables.

\item[{Return type}] \leavevmode
dict

\end{description}\end{quote}

\end{fulllineitems}

\index{logic\_cal\_accfs() (in module envlib.extract\_data)@\spxentry{logic\_cal\_accfs()}\spxextra{in module envlib.extract\_data}}

\begin{fulllineitems}
\phantomsection\label{\detokenize{envlib:envlib.extract_data.logic_cal_accfs}}\pysiglinewithargsret{\sphinxcode{\sphinxupquote{envlib.extract\_data.}}\sphinxbfcode{\sphinxupquote{logic\_cal\_accfs}}}{\emph{\DUrole{n}{variables}}}{}
Build a dictionary containing logical valules correspond to the possibility to calculate of aCCF.
\begin{quote}\begin{description}
\item[{Parameters}] \leavevmode
\sphinxstyleliteralstrong{\sphinxupquote{variables}} (\sphinxstyleliteralemphasis{\sphinxupquote{dict}}) \textendash{} variables available in the given dataset.

\item[{Returns}] \leavevmode
Dictionary containing logical valules correspond to the possibility to calculate of aCCFs.

\item[{Return type}] \leavevmode
dict

\end{description}\end{quote}

\end{fulllineitems}



\subsubsection{envlib.io module}
\label{\detokenize{envlib:module-envlib.io}}\label{\detokenize{envlib:envlib-io-module}}\index{module@\spxentry{module}!envlib.io@\spxentry{envlib.io}}\index{envlib.io@\spxentry{envlib.io}!module@\spxentry{module}}

\subsubsection{envlib.main\_processing module}
\label{\detokenize{envlib:module-envlib.main_processing}}\label{\detokenize{envlib:envlib-main-processing-module}}\index{module@\spxentry{module}!envlib.main\_processing@\spxentry{envlib.main\_processing}}\index{envlib.main\_processing@\spxentry{envlib.main\_processing}!module@\spxentry{module}}\index{ClimateImpact (class in envlib.main\_processing)@\spxentry{ClimateImpact}\spxextra{class in envlib.main\_processing}}

\begin{fulllineitems}
\phantomsection\label{\detokenize{envlib:envlib.main_processing.ClimateImpact}}\pysiglinewithargsret{\sphinxbfcode{\sphinxupquote{class }}\sphinxcode{\sphinxupquote{envlib.main\_processing.}}\sphinxbfcode{\sphinxupquote{ClimateImpact}}}{\emph{\DUrole{n}{path}}, \emph{\DUrole{o}{**}\DUrole{n}{problem\_config}}}{}
Bases: \sphinxcode{\sphinxupquote{object}}
\index{auto\_plotting() (envlib.main\_processing.ClimateImpact method)@\spxentry{auto\_plotting()}\spxextra{envlib.main\_processing.ClimateImpact method}}

\begin{fulllineitems}
\phantomsection\label{\detokenize{envlib:envlib.main_processing.ClimateImpact.auto_plotting}}\pysiglinewithargsret{\sphinxbfcode{\sphinxupquote{auto\_plotting}}}{}{}
\end{fulllineitems}

\index{calculate\_accfs() (envlib.main\_processing.ClimateImpact method)@\spxentry{calculate\_accfs()}\spxextra{envlib.main\_processing.ClimateImpact method}}

\begin{fulllineitems}
\phantomsection\label{\detokenize{envlib:envlib.main_processing.ClimateImpact.calculate_accfs}}\pysiglinewithargsret{\sphinxbfcode{\sphinxupquote{calculate\_accfs}}}{\emph{\DUrole{o}{**}\DUrole{n}{seetings}}}{}
\end{fulllineitems}

\index{generate\_output() (envlib.main\_processing.ClimateImpact method)@\spxentry{generate\_output()}\spxextra{envlib.main\_processing.ClimateImpact method}}

\begin{fulllineitems}
\phantomsection\label{\detokenize{envlib:envlib.main_processing.ClimateImpact.generate_output}}\pysiglinewithargsret{\sphinxbfcode{\sphinxupquote{generate\_output}}}{}{}
\end{fulllineitems}


\end{fulllineitems}



\subsubsection{envlib.processing\_surf\_vars module}
\label{\detokenize{envlib:module-envlib.processing_surf_vars}}\label{\detokenize{envlib:envlib-processing-surf-vars-module}}\index{module@\spxentry{module}!envlib.processing\_surf\_vars@\spxentry{envlib.processing\_surf\_vars}}\index{envlib.processing\_surf\_vars@\spxentry{envlib.processing\_surf\_vars}!module@\spxentry{module}}\index{extend\_olr\_pl\_4d() (in module envlib.processing\_surf\_vars)@\spxentry{extend\_olr\_pl\_4d()}\spxextra{in module envlib.processing\_surf\_vars}}

\begin{fulllineitems}
\phantomsection\label{\detokenize{envlib:envlib.processing_surf_vars.extend_olr_pl_4d}}\pysiglinewithargsret{\sphinxcode{\sphinxupquote{envlib.processing\_surf\_vars.}}\sphinxbfcode{\sphinxupquote{extend\_olr\_pl\_4d}}}{\emph{\DUrole{n}{sur\_var}}, \emph{\DUrole{n}{pl\_var}}, \emph{\DUrole{n}{index}}, \emph{\DUrole{n}{fore\_step}}}{}
Calculate outgoing longwave radiation (OLR) {[}W/m2{]} at TOA from the parameter, top net thermal radiation (ttr)
{[}J/m2{]}, and repeat it for to pressure levels for the sake of consistency of dimensions. For a specific time
regarding inputted index, OLR is calculated in 3D (i.e, level, latitude, longitude).
\begin{quote}\begin{description}
\item[{Parameters}] \leavevmode\begin{itemize}
\item {} 
\sphinxstyleliteralstrong{\sphinxupquote{sur\_var}} (\sphinxstyleliteralemphasis{\sphinxupquote{Dataset}}) \textendash{} Dataset containing surface parameters openned with xarray.

\item {} 
\sphinxstyleliteralstrong{\sphinxupquote{pl\_var}} (\sphinxstyleliteralemphasis{\sphinxupquote{Dataset}}) \textendash{} Dataset containing pressure level parameters openned with xarray.

\item {} 
\sphinxstyleliteralstrong{\sphinxupquote{index}} (\sphinxstyleliteralemphasis{\sphinxupquote{int}}) \textendash{} Index of the time that exist in the dataset of pressure level parameters at this step.

\item {} 
\sphinxstyleliteralstrong{\sphinxupquote{fore\_step}} (\sphinxstyleliteralemphasis{\sphinxupquote{int}}) \textendash{} Forecast step in hours.

\end{itemize}

\item[{Returns arr}] \leavevmode
OLR with 3D dimensiones (i.e., level, latitude, longitude).

\item[{Return type}] \leavevmode
array

\end{description}\end{quote}

\end{fulllineitems}

\index{extend\_olr\_pl\_5d() (in module envlib.processing\_surf\_vars)@\spxentry{extend\_olr\_pl\_5d()}\spxextra{in module envlib.processing\_surf\_vars}}

\begin{fulllineitems}
\phantomsection\label{\detokenize{envlib:envlib.processing_surf_vars.extend_olr_pl_5d}}\pysiglinewithargsret{\sphinxcode{\sphinxupquote{envlib.processing\_surf\_vars.}}\sphinxbfcode{\sphinxupquote{extend\_olr\_pl\_5d}}}{\emph{\DUrole{n}{sur\_var}}, \emph{\DUrole{n}{pl\_var}}, \emph{\DUrole{n}{index}}, \emph{\DUrole{n}{fore\_step}}}{}
Calculate outgoing longwave radiation (OLR) {[}W/m2{]} at TOA from the parameter, top net thermal radiation (ttr)
{[}J/m2{]}, and repeat it for to pressure levels for the sake of consistency of dimensions. For a specific time
regarding inputted index, OLR is calculated in 4D (i.e, number, level, latitude, longitude).
\begin{quote}\begin{description}
\item[{Parameters}] \leavevmode\begin{itemize}
\item {} 
\sphinxstyleliteralstrong{\sphinxupquote{sur\_var}} (\sphinxstyleliteralemphasis{\sphinxupquote{Dataset}}) \textendash{} Dataset containing surface parameters openned with xarray.

\item {} 
\sphinxstyleliteralstrong{\sphinxupquote{pl\_var}} (\sphinxstyleliteralemphasis{\sphinxupquote{Dataset}}) \textendash{} Dataset containing pressure level parameters openned with xarray.

\item {} 
\sphinxstyleliteralstrong{\sphinxupquote{index}} (\sphinxstyleliteralemphasis{\sphinxupquote{int}}) \textendash{} Index of the time that exist in the dataset of pressure level parameters at this step.

\item {} 
\sphinxstyleliteralstrong{\sphinxupquote{fore\_step}} (\sphinxstyleliteralemphasis{\sphinxupquote{int}}) \textendash{} Forecast step in hours.

\end{itemize}

\item[{Returns arr}] \leavevmode
OLR with 4D dimensiones (i.e., number, level, latitude, longitude).

\item[{Return type}] \leavevmode
array

\end{description}\end{quote}

\end{fulllineitems}

\index{get\_olr() (in module envlib.processing\_surf\_vars)@\spxentry{get\_olr()}\spxextra{in module envlib.processing\_surf\_vars}}

\begin{fulllineitems}
\phantomsection\label{\detokenize{envlib:envlib.processing_surf_vars.get_olr}}\pysiglinewithargsret{\sphinxcode{\sphinxupquote{envlib.processing\_surf\_vars.}}\sphinxbfcode{\sphinxupquote{get\_olr}}}{\emph{\DUrole{n}{sur\_var}}, \emph{\DUrole{n}{pl\_var}}, \emph{\DUrole{n}{number}\DUrole{o}{=}\DUrole{default_value}{True}}, \emph{\DUrole{n}{fore\_step}\DUrole{o}{=}\DUrole{default_value}{None}}}{}
Calculate outgoing longwave radiation (OLR) {[}W/m2{]} at TOA from the parameter, top net thermal radiation (ttr)
{[}J/m2{]}. OLR is calculated in 5D or 4D depending on the existance of ensemble members.
\begin{quote}\begin{description}
\item[{Parameters}] \leavevmode\begin{itemize}
\item {} 
\sphinxstyleliteralstrong{\sphinxupquote{sur\_var}} (\sphinxstyleliteralemphasis{\sphinxupquote{Dataset}}) \textendash{} Dataset containing surface parameters openned with xarray.

\item {} 
\sphinxstyleliteralstrong{\sphinxupquote{pl\_var}} (\sphinxstyleliteralemphasis{\sphinxupquote{int}}) \textendash{} Dataset containing pressure level parameters openned with xarray.

\item {} 
\sphinxstyleliteralstrong{\sphinxupquote{number}} (\sphinxstyleliteralemphasis{\sphinxupquote{bool}}) \textendash{} Determines whether the weather data contains ensemble members or not.

\item {} 
\sphinxstyleliteralstrong{\sphinxupquote{fore\_step}} \textendash{} Forecast step in hours.

\end{itemize}

\item[{Returns arr}] \leavevmode
OLR.

\item[{Return type}] \leavevmode
numpy.ndarray

\end{description}\end{quote}

\end{fulllineitems}

\index{get\_olr\_4d() (in module envlib.processing\_surf\_vars)@\spxentry{get\_olr\_4d()}\spxextra{in module envlib.processing\_surf\_vars}}

\begin{fulllineitems}
\phantomsection\label{\detokenize{envlib:envlib.processing_surf_vars.get_olr_4d}}\pysiglinewithargsret{\sphinxcode{\sphinxupquote{envlib.processing\_surf\_vars.}}\sphinxbfcode{\sphinxupquote{get\_olr\_4d}}}{\emph{\DUrole{n}{sur\_var}}, \emph{\DUrole{n}{pl\_var}}, \emph{\DUrole{n}{thr}}, \emph{\DUrole{n}{fore\_step}\DUrole{o}{=}\DUrole{default_value}{None}}}{}
Calculate outgoing longwave radiation (OLR) {[}W/m2{]} at TOA from the parameter, top net thermal radiation (ttr)
{[}J/m2{]}. OLR is calculated in 4D (i.e, time, level, latitude, longitude).
\begin{quote}\begin{description}
\item[{Parameters}] \leavevmode\begin{itemize}
\item {} 
\sphinxstyleliteralstrong{\sphinxupquote{sur\_var}} (\sphinxstyleliteralemphasis{\sphinxupquote{Dataset}}) \textendash{} Dataset containing surface parameters openned with xarray.

\item {} 
\sphinxstyleliteralstrong{\sphinxupquote{pl\_var}} (\sphinxstyleliteralemphasis{\sphinxupquote{int}}) \textendash{} Dataset containing pressure level parameters openned with xarray.

\item {} 
\sphinxstyleliteralstrong{\sphinxupquote{thr}} (\sphinxstyleliteralemphasis{\sphinxupquote{dict}}) \textendash{} Thresholds to automatically determine forecast steps.

\item {} 
\sphinxstyleliteralstrong{\sphinxupquote{fore\_step}} \textendash{} Forecast step in hours.

\end{itemize}

\item[{Returns arr}] \leavevmode
OLR with 4D dimensiones (i.e., time, level, latitude, longitude).

\item[{Return type}] \leavevmode
numpy.ndarray

\end{description}\end{quote}

\end{fulllineitems}

\index{get\_olr\_5d() (in module envlib.processing\_surf\_vars)@\spxentry{get\_olr\_5d()}\spxextra{in module envlib.processing\_surf\_vars}}

\begin{fulllineitems}
\phantomsection\label{\detokenize{envlib:envlib.processing_surf_vars.get_olr_5d}}\pysiglinewithargsret{\sphinxcode{\sphinxupquote{envlib.processing\_surf\_vars.}}\sphinxbfcode{\sphinxupquote{get\_olr\_5d}}}{\emph{\DUrole{n}{sur\_var}}, \emph{\DUrole{n}{pl\_var}}, \emph{\DUrole{n}{thr}}, \emph{\DUrole{n}{fore\_step}\DUrole{o}{=}\DUrole{default_value}{None}}}{}
Calculate outgoing longwave radiation (OLR) {[}W/m2{]} at TOA from the parameter, top net thermal radiation (ttr)
{[}J/m2{]}. OLR is calculated in 5D (i.e, time, number, level, latitude, longitude).
\begin{quote}\begin{description}
\item[{Parameters}] \leavevmode\begin{itemize}
\item {} 
\sphinxstyleliteralstrong{\sphinxupquote{sur\_var}} (\sphinxstyleliteralemphasis{\sphinxupquote{Dataset}}) \textendash{} Dataset containing surface parameters openned with xarray.

\item {} 
\sphinxstyleliteralstrong{\sphinxupquote{pl\_var}} (\sphinxstyleliteralemphasis{\sphinxupquote{int}}) \textendash{} Dataset containing pressure level parameters openned with xarray.

\item {} 
\sphinxstyleliteralstrong{\sphinxupquote{thr}} (\sphinxstyleliteralemphasis{\sphinxupquote{dict}}) \textendash{} Thresholds to automatically determine forecast steps.

\item {} 
\sphinxstyleliteralstrong{\sphinxupquote{fore\_step}} \textendash{} Forecast step in hours.

\end{itemize}

\item[{Returns arr}] \leavevmode
OLR with 5D dimensiones (i.e., time, number, level, latitude, longitude).

\item[{Return type}] \leavevmode
numpy.ndarray

\end{description}\end{quote}

\end{fulllineitems}



\subsubsection{envlib.weather\_store module}
\label{\detokenize{envlib:module-envlib.weather_store}}\label{\detokenize{envlib:envlib-weather-store-module}}\index{module@\spxentry{module}!envlib.weather\_store@\spxentry{envlib.weather\_store}}\index{envlib.weather\_store@\spxentry{envlib.weather\_store}!module@\spxentry{module}}\index{GeoArrayHandler (class in envlib.weather\_store)@\spxentry{GeoArrayHandler}\spxextra{class in envlib.weather\_store}}

\begin{fulllineitems}
\phantomsection\label{\detokenize{envlib:envlib.weather_store.GeoArrayHandler}}\pysigline{\sphinxbfcode{\sphinxupquote{class }}\sphinxcode{\sphinxupquote{envlib.weather\_store.}}\sphinxbfcode{\sphinxupquote{GeoArrayHandler}}}
Bases: \sphinxcode{\sphinxupquote{object}}
\index{axes (envlib.weather\_store.GeoArrayHandler attribute)@\spxentry{axes}\spxextra{envlib.weather\_store.GeoArrayHandler attribute}}

\begin{fulllineitems}
\phantomsection\label{\detokenize{envlib:envlib.weather_store.GeoArrayHandler.axes}}\pysigline{\sphinxbfcode{\sphinxupquote{axes}}\sphinxbfcode{\sphinxupquote{: dict}}}
\end{fulllineitems}

\index{bitriangular\_filter (envlib.weather\_store.GeoArrayHandler attribute)@\spxentry{bitriangular\_filter}\spxextra{envlib.weather\_store.GeoArrayHandler attribute}}

\begin{fulllineitems}
\phantomsection\label{\detokenize{envlib:envlib.weather_store.GeoArrayHandler.bitriangular_filter}}\pysigline{\sphinxbfcode{\sphinxupquote{bitriangular\_filter}}\sphinxbfcode{\sphinxupquote{ = array({[}{[}0.0625, 0.125 , 0.0625{]},        {[}0.125 , 0.25  , 0.125 {]},        {[}0.0625, 0.125 , 0.0625{]}{]})}}}
\end{fulllineitems}

\index{decimate() (envlib.weather\_store.GeoArrayHandler method)@\spxentry{decimate()}\spxextra{envlib.weather\_store.GeoArrayHandler method}}

\begin{fulllineitems}
\phantomsection\label{\detokenize{envlib:envlib.weather_store.GeoArrayHandler.decimate}}\pysiglinewithargsret{\sphinxbfcode{\sphinxupquote{decimate}}}{\emph{\DUrole{n}{array}}}{}
\end{fulllineitems}

\index{decimate\_3d() (envlib.weather\_store.GeoArrayHandler method)@\spxentry{decimate\_3d()}\spxextra{envlib.weather\_store.GeoArrayHandler method}}

\begin{fulllineitems}
\phantomsection\label{\detokenize{envlib:envlib.weather_store.GeoArrayHandler.decimate_3d}}\pysiglinewithargsret{\sphinxbfcode{\sphinxupquote{decimate\_3d}}}{\emph{\DUrole{n}{array}}}{}
\end{fulllineitems}

\index{decimate\_4d() (envlib.weather\_store.GeoArrayHandler method)@\spxentry{decimate\_4d()}\spxextra{envlib.weather\_store.GeoArrayHandler method}}

\begin{fulllineitems}
\phantomsection\label{\detokenize{envlib:envlib.weather_store.GeoArrayHandler.decimate_4d}}\pysiglinewithargsret{\sphinxbfcode{\sphinxupquote{decimate\_4d}}}{\emph{\DUrole{n}{array}}}{}
\end{fulllineitems}

\index{decimate\_5d() (envlib.weather\_store.GeoArrayHandler method)@\spxentry{decimate\_5d()}\spxextra{envlib.weather\_store.GeoArrayHandler method}}

\begin{fulllineitems}
\phantomsection\label{\detokenize{envlib:envlib.weather_store.GeoArrayHandler.decimate_5d}}\pysiglinewithargsret{\sphinxbfcode{\sphinxupquote{decimate\_5d}}}{\emph{\DUrole{n}{array}}}{}
\end{fulllineitems}

\index{down2() (envlib.weather\_store.GeoArrayHandler method)@\spxentry{down2()}\spxextra{envlib.weather\_store.GeoArrayHandler method}}

\begin{fulllineitems}
\phantomsection\label{\detokenize{envlib:envlib.weather_store.GeoArrayHandler.down2}}\pysiglinewithargsret{\sphinxbfcode{\sphinxupquote{down2}}}{\emph{\DUrole{n}{array}}}{}
Decimates a 2D array by a factor of two after applying a triangular filter

\end{fulllineitems}

\index{down2\_coord() (envlib.weather\_store.GeoArrayHandler class method)@\spxentry{down2\_coord()}\spxextra{envlib.weather\_store.GeoArrayHandler class method}}

\begin{fulllineitems}
\phantomsection\label{\detokenize{envlib:envlib.weather_store.GeoArrayHandler.down2_coord}}\pysiglinewithargsret{\sphinxbfcode{\sphinxupquote{classmethod }}\sphinxbfcode{\sphinxupquote{down2\_coord}}}{\emph{\DUrole{n}{array}}}{}
\end{fulllineitems}

\index{get\_coords() (envlib.weather\_store.GeoArrayHandler method)@\spxentry{get\_coords()}\spxextra{envlib.weather\_store.GeoArrayHandler method}}

\begin{fulllineitems}
\phantomsection\label{\detokenize{envlib:envlib.weather_store.GeoArrayHandler.get_coords}}\pysiglinewithargsret{\sphinxbfcode{\sphinxupquote{get\_coords}}}{}{}
\end{fulllineitems}

\index{triangular\_filter (envlib.weather\_store.GeoArrayHandler attribute)@\spxentry{triangular\_filter}\spxextra{envlib.weather\_store.GeoArrayHandler attribute}}

\begin{fulllineitems}
\phantomsection\label{\detokenize{envlib:envlib.weather_store.GeoArrayHandler.triangular_filter}}\pysigline{\sphinxbfcode{\sphinxupquote{triangular\_filter}}\sphinxbfcode{\sphinxupquote{ = array({[}0.25, 0.5 , 0.25{]})}}}
\end{fulllineitems}


\end{fulllineitems}

\index{WeatherStore (class in envlib.weather\_store)@\spxentry{WeatherStore}\spxextra{class in envlib.weather\_store}}

\begin{fulllineitems}
\phantomsection\label{\detokenize{envlib:envlib.weather_store.WeatherStore}}\pysiglinewithargsret{\sphinxbfcode{\sphinxupquote{class }}\sphinxcode{\sphinxupquote{envlib.weather\_store.}}\sphinxbfcode{\sphinxupquote{WeatherStore}}}{\emph{\DUrole{n}{weather\_data}}, \emph{\DUrole{n}{weather\_data\_sur}\DUrole{o}{=}\DUrole{default_value}{None}}, \emph{\DUrole{n}{flipud}\DUrole{o}{=}\DUrole{default_value}{\textquotesingle{}auto\textquotesingle{}}}, \emph{\DUrole{o}{**}\DUrole{n}{weather\_config}}}{}
Bases: {\hyperref[\detokenize{envlib:envlib.weather_store.WeatherStore_}]{\sphinxcrossref{\sphinxcode{\sphinxupquote{envlib.weather\_store.WeatherStore\_}}}}}

Processing weather data
\index{axes (envlib.weather\_store.WeatherStore attribute)@\spxentry{axes}\spxextra{envlib.weather\_store.WeatherStore attribute}}

\begin{fulllineitems}
\phantomsection\label{\detokenize{envlib:envlib.weather_store.WeatherStore.axes}}\pysigline{\sphinxbfcode{\sphinxupquote{axes}}}
\end{fulllineitems}

\index{get\_xarray() (envlib.weather\_store.WeatherStore method)@\spxentry{get\_xarray()}\spxextra{envlib.weather\_store.WeatherStore method}}

\begin{fulllineitems}
\phantomsection\label{\detokenize{envlib:envlib.weather_store.WeatherStore.get_xarray}}\pysiglinewithargsret{\sphinxbfcode{\sphinxupquote{get\_xarray}}}{}{}
Build xarray dataset.
\begin{quote}\begin{description}
\item[{Returns ds}] \leavevmode
xarray dataset containing user\sphinxhyphen{}difned variables (e.g., merged aCCFs, mean aCCFs, Climate hotspots).

\item[{Return type}] \leavevmode
dataset

\end{description}\end{quote}

\end{fulllineitems}

\index{reduce\_domain() (envlib.weather\_store.WeatherStore method)@\spxentry{reduce\_domain()}\spxextra{envlib.weather\_store.WeatherStore method}}

\begin{fulllineitems}
\phantomsection\label{\detokenize{envlib:envlib.weather_store.WeatherStore.reduce_domain}}\pysiglinewithargsret{\sphinxbfcode{\sphinxupquote{reduce\_domain}}}{\emph{\DUrole{n}{bounds}}, \emph{\DUrole{n}{verbose}\DUrole{o}{=}\DUrole{default_value}{False}}}{}
Reduces horizontal domain and time .
\begin{quote}\begin{description}
\item[{Returns bounds}] \leavevmode
ranges defined as tuple (e.g., lat\_bound=(35, 60.0)).

\item[{Return type}] \leavevmode
dict

\end{description}\end{quote}

\end{fulllineitems}


\end{fulllineitems}

\index{WeatherStore\_ (class in envlib.weather\_store)@\spxentry{WeatherStore\_}\spxextra{class in envlib.weather\_store}}

\begin{fulllineitems}
\phantomsection\label{\detokenize{envlib:envlib.weather_store.WeatherStore_}}\pysigline{\sphinxbfcode{\sphinxupquote{class }}\sphinxcode{\sphinxupquote{envlib.weather\_store.}}\sphinxbfcode{\sphinxupquote{WeatherStore\_}}}
Bases: {\hyperref[\detokenize{envlib:envlib.weather_store.GeoArrayHandler}]{\sphinxcrossref{\sphinxcode{\sphinxupquote{envlib.weather\_store.GeoArrayHandler}}}}}
\index{axes (envlib.weather\_store.WeatherStore\_ attribute)@\spxentry{axes}\spxextra{envlib.weather\_store.WeatherStore\_ attribute}}

\begin{fulllineitems}
\phantomsection\label{\detokenize{envlib:envlib.weather_store.WeatherStore_.axes}}\pysigline{\sphinxbfcode{\sphinxupquote{axes}}}
\end{fulllineitems}


\end{fulllineitems}

\index{get\_bound\_indexes() (in module envlib.weather\_store)@\spxentry{get\_bound\_indexes()}\spxextra{in module envlib.weather\_store}}

\begin{fulllineitems}
\phantomsection\label{\detokenize{envlib:envlib.weather_store.get_bound_indexes}}\pysiglinewithargsret{\sphinxcode{\sphinxupquote{envlib.weather\_store.}}\sphinxbfcode{\sphinxupquote{get\_bound\_indexes}}}{\emph{\DUrole{n}{arr}}, \emph{\DUrole{n}{bounds}}, \emph{\DUrole{n}{verbose}\DUrole{o}{=}\DUrole{default_value}{False}}}{}
Determine indices of a given array (e.g., latitude and longitude) needed for cutting geographical areas with respect to the user\sphinxhyphen{}difined bounds.
\begin{quote}\begin{description}
\item[{Parameters}] \leavevmode\begin{itemize}
\item {} 
\sphinxstyleliteralstrong{\sphinxupquote{arr}} (\sphinxstyleliteralemphasis{\sphinxupquote{numpy.ndarray}}) \textendash{} a given array (e.g., latitude and longitude).

\item {} 
\sphinxstyleliteralstrong{\sphinxupquote{bounds}} (\sphinxstyleliteralemphasis{\sphinxupquote{tuple}}) \textendash{} user\sphinxhyphen{}defined bounds.

\item {} 
\sphinxstyleliteralstrong{\sphinxupquote{verbose}} (\sphinxstyleliteralemphasis{\sphinxupquote{bool}}) \textendash{} Show determined indices.

\end{itemize}

\item[{Returns slice(low, high)}] \leavevmode
return the determined low and high indices of the given array that includes the

\end{description}\end{quote}

defined bounds.
:rtype: slice

\end{fulllineitems}



\subsubsection{Module contents}
\label{\detokenize{envlib:module-envlib}}\label{\detokenize{envlib:module-contents}}\index{module@\spxentry{module}!envlib@\spxentry{envlib}}\index{envlib@\spxentry{envlib}!module@\spxentry{module}}

\subsection{setup module}
\label{\detokenize{setup:setup-module}}\label{\detokenize{setup::doc}}

\subsection{test package}
\label{\detokenize{test:test-package}}\label{\detokenize{test::doc}}

\subsubsection{Submodules}
\label{\detokenize{test:submodules}}

\subsubsection{test.test\_main module}
\label{\detokenize{test:module-test.test_main}}\label{\detokenize{test:test-test-main-module}}\index{module@\spxentry{module}!test.test\_main@\spxentry{test.test\_main}}\index{test.test\_main@\spxentry{test.test\_main}!module@\spxentry{module}}\index{test\_main() (in module test.test\_main)@\spxentry{test\_main()}\spxextra{in module test.test\_main}}

\begin{fulllineitems}
\phantomsection\label{\detokenize{test:test.test_main.test_main}}\pysiglinewithargsret{\sphinxcode{\sphinxupquote{test.test\_main.}}\sphinxbfcode{\sphinxupquote{test\_main}}}{}{}
\end{fulllineitems}



\subsubsection{Module contents}
\label{\detokenize{test:module-test}}\label{\detokenize{test:module-contents}}\index{module@\spxentry{module}!test@\spxentry{test}}\index{test@\spxentry{test}!module@\spxentry{module}}

\chapter{Indices and tables}
\label{\detokenize{index:indices-and-tables}}\begin{itemize}
\item {} 
\DUrole{xref,std,std-ref}{genindex}

\item {} 
\DUrole{xref,std,std-ref}{modindex}

\item {} 
\DUrole{xref,std,std-ref}{search}

\end{itemize}


\section{Acknowledmgements}
\label{\detokenize{index:acknowledmgements}}
\noindent{\hspace*{\fill}\sphinxincludegraphics[width=100\sphinxpxdimen]{{Alarm_LOGO}.eps}\hspace*{\fill}}

\sphinxstyleemphasis{This library has been developed within \textbf{ALARM} and \textbf{FLyATM4E} Projects. }
\begin{itemize}
\item 
\textbf{ALARM} has received funding from the SESAR Joint Undertaking (JU) under grant agreement No 891467. The JU receives support from the European Union’s Horizon 2020 research and innovation programme and the SESAR JU members other than the Union.
\item
\textbf{FLyATM4E} has received funding from the SESAR Joint Undertaking under the European Union’s Horizon 2020 research and innovation programme under grant agreement No 891317. The JU receives support from the European Union’s Horizon 2020 research and innovation programme and the SESAR JU members other than the Union.
\end{itemize}
\begin{quote}


\begin{savenotes}\sphinxattablestart
\centering
\begin{tabulary}{\linewidth}[t]{|T|T|}
\hline

\sphinxincludegraphics[width=50\sphinxpxdimen]{{european-union_flag_yellow_high}.jpg}
&
\sphinxincludegraphics[width=50\sphinxpxdimen]{{sesar}.png}
\\
\hline
\end{tabulary}
\par
\sphinxattableend\end{savenotes}
\end{quote}


\renewcommand{\indexname}{Python Module Index}
\begin{sphinxtheindex}
\let\bigletter\sphinxstyleindexlettergroup
\bigletter{e}
\item\relax\sphinxstyleindexentry{envlib}\sphinxstyleindexpageref{envlib:\detokenize{module-envlib}}
\item\relax\sphinxstyleindexentry{envlib.accf}\sphinxstyleindexpageref{envlib:\detokenize{module-envlib.accf}}
\item\relax\sphinxstyleindexentry{envlib.calc\_altrv\_vars}\sphinxstyleindexpageref{envlib:\detokenize{module-envlib.calc_altrv_vars}}
\item\relax\sphinxstyleindexentry{envlib.contrail}\sphinxstyleindexpageref{envlib:\detokenize{module-envlib.contrail}}
\item\relax\sphinxstyleindexentry{envlib.database}\sphinxstyleindexpageref{envlib:\detokenize{module-envlib.database}}
\item\relax\sphinxstyleindexentry{envlib.emission\_indices}\sphinxstyleindexpageref{envlib:\detokenize{module-envlib.emission_indices}}
\item\relax\sphinxstyleindexentry{envlib.extend\_dim}\sphinxstyleindexpageref{envlib:\detokenize{module-envlib.extend_dim}}
\item\relax\sphinxstyleindexentry{envlib.extract\_data}\sphinxstyleindexpageref{envlib:\detokenize{module-envlib.extract_data}}
\item\relax\sphinxstyleindexentry{envlib.io}\sphinxstyleindexpageref{envlib:\detokenize{module-envlib.io}}
\item\relax\sphinxstyleindexentry{envlib.main\_processing}\sphinxstyleindexpageref{envlib:\detokenize{module-envlib.main_processing}}
\item\relax\sphinxstyleindexentry{envlib.processing\_surf\_vars}\sphinxstyleindexpageref{envlib:\detokenize{module-envlib.processing_surf_vars}}
\item\relax\sphinxstyleindexentry{envlib.weather\_store}\sphinxstyleindexpageref{envlib:\detokenize{module-envlib.weather_store}}
\indexspace
\bigletter{t}
\item\relax\sphinxstyleindexentry{test}\sphinxstyleindexpageref{test:\detokenize{module-test}}
\item\relax\sphinxstyleindexentry{test.test\_main}\sphinxstyleindexpageref{test:\detokenize{module-test.test_main}}
\end{sphinxtheindex}

\renewcommand{\indexname}{Index}
\printindex
\end{document}